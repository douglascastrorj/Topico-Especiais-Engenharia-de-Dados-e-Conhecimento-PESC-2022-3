\section{Revisão Informal}

Nesta seção, serão discutidos alguns trabalhos relacionados ao tema de
categorização fake news de modo a entender quais são abordagens que estão sendo
propostas na literatura como também a qualidade dos resultados obtidos por estes.

\subsection{\textit{A Survey on Natural Language Processing for Fake News Detection}}

Em \cite{oshikawa2020} os autores conduziram uma pesquisa sobre processamento de linguagem natural para detecção de notícias falsas em que eles forneceram uma visão geral dos esforços de pesquisa para detecção de notícias falsas e uma comparação sistemática de suas definições de tarefas, conjuntos de dados, construção de modelos, e performances. Em suma, suas contribuições gerias foram as seguintes: \\

\begin{itemize}
    \item Forneceram uma revisão das soluções de processamento de linguagem natural para detecção atumática de notícias falsas.
    
    \item Analisaram sistematicamente como a detecção de notícias falsas estão alinhadas com as tarefas de \textit{NLP} existentes e discutiram as suposições e questões notáveis para diferentes fórmulações do problema;
    
    \item Categorizaram e resumiram os \textit{datasets} disponíveis, abordagens de \textit{NLP} e resultados, fornecendo experiências em primeira mão e introduções acessíveis para novos pesquisadores interessado neste problema.
\end{itemize}

\subsubsection{Problemas Relacionados a detecção de \textit{fake news}}

A respeito de problemas relacionados a detecção automática de \textit{fake news}, os autores destacaram 4 que são eles: \\

\textbf{\textit{Fact-Checking}} Em geral, detecção de \textit{fake news} usualmente foca em eventos novos enquanto \textit{fact-checking} é mais amplo. Vale resaltar que, segundo os autores, muitos pesquisadores não distinguem detecção de \textit{fake news} e \textit{fact-checking} pois ambos visam avaliar a veracidade de alegações;  \\

\textbf{\textit{Rumor Detection}} Não há uma definição consistente de detecção de rumor. Uma pesquisa recente (\cite{zubiaga2018}) define detecção de boatos como separar declarações pessoais em boatos ou não boatos, onde o boato é definido como uma declaração que consiste em informações não verificadas no momento da postagem. Dentro de outras palavras, o boato deve conter informações que possam ser verificados em vez de opiniões ou sentimentos subjetivos; \\


\textbf{\textit{Stance Detection}} é definida como a tarefa de aferir de qual lado de um debate um autor está no texto. Isto difere de detecção de \textit{fake news} no sentido que não se trata sobre a veracidade mas da consistência. \textit{Stance Detection} pode ser uma subtarefa de detecção de \textit{fake news} pois ela pode ser aplicada para buscar documentos para evidência; \\


\textbf{\textit{Sentiment Analysis}} A análise de sentimentos é a tarefa de extrair emoções, como impressão favorável ou desfavorável dos clientes de um restaurante. Diferente da detecção de boatos e de notícias falsas, a análise de sentimento não é fazer uma verificação objetiva de alegações, mas para analisar as emoções pessoais. \\

