\section{Proposta}

Este trabalho ataca o problema da detecção de \textit{fake news} binário \cite{}, isto e, existem apenas duas categorias: Falsas e Verdadeiras. Para isto sera utilizado o BERT \cite{}, que e uma rede neural desenvolvida pela Google capaz de aprender as formas de expressões da linguagem humana e e baseada em um modelo de \textit{NLP} chamado \textit{Transformer}, que consegue entender os relacionamentos entre palavras em uma frase em vez de apenas visualiza-las uma por uma em ordem.


Neste trabalho será utilizado o \textit{dataset} em português \textit{Fake Corpus Br} proposto por \citet{Silva2020} e em seguida será executada uma comparação entre os resultados obtidos por esse trabalho e o proposto neste artigo.

% Aqui entram a descrição detalhada do problema sendo tratado e o plano de Solução.

% Aqui será explicada a sua proposta, com cuidado para atingir o objetivo descrito na introdução (ele pode ser lembrado) e como é diferente dos trabalhos correlatos.





