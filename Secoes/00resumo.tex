\begin{abstract}
   A disseminação de notícias falsas representa uma ameaça real à confiabilidade das informações. 
      O alcance das plataformas de comunicação como Twitter, Facebook, Youtube e afins contribui para a disseminação rápida desse tipo de conteúdo. 
   Portanto, a detecção dessas notícias tem atraído  atenção, dado que sua identificação pode beneficiar a sociedade em diversos aspectos, como por exemplo na política e no mercado financeiro. 
   Porém, informações presentes em uma notícia pode ter vários graus de veracidade, isto é,  podem ser parcialmente verdadeiras ou parcialmente falsas.
   Isso torna a resolução do problema de detecção e \textit{fake news} bastante desafiadora.
   Neste trabalho, é feita uma breve revisão da literatura a fim de identificar o estado da arte de detecção de notícias falsas, mais especificamente: quais são as abordagens mais utilizadas em termos de modelagem do problema; quais as técnicas de processamento de linguagem natural e os algoritmos de aprendizado de máquina que são mais eficazes para esta tarefa. Por fim, são feitos experimentos utilizando a rede neural \textbf{BERT}, que são comparados com os métodos utilizados na revisão da literatura.
   
   
%     A maioria das publicações usa um resumo não estruturado, que será usado aqui, mas em todo caso, nós usaremos os mesmos tópicos propostos em uma estrutura chamada IMRD~\citep{cmucs}. O artigo, porém, não usará essa estrutura, mas a que aqui está descrita.
%     O resumo deve ser construído de forma a fornecer as seguintes informações: 
%     \begin{itemize}
%         \item introdução, dizendo o que é o artigo
%         \item métodos
%         \item resultados principais
%         \item uma discussão dos resultados, por exemplo, porque é importante
%     \end{itemize} 


\end{abstract}
