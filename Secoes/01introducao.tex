\section{Introdução}

Atualmente, dado o alcance das plataformas de comunicação como Twitter, Facebook, Youtube e afins, disseminar um conhecimento ou uma informação torna-se instantâneo, bastando apenas um clique. Tal fato acaba se tornando um problema que contribui para a disseminação rápida e desenfreada de notícias falsas e outras formas de desinformação. Dado este fato, este trabalho almeja atacar o problema da ``Detecção de \textit{Fake News}'' de modo a avaliar diferentes algorítimos de aprendizado de máquina e técnicas de \textit{NLP - \textit{Natural Language Processing}}. \\

% Com o avanço da tecnologia, torna-se cada vez mais fácil falsificar áudios e vídeos, levando as \textit{fake} news a uma nova e mais preocupante fase conhecida como \emph{Deepfake} que é uma técnica de síntese de imagens humanas baseadas em inteligência  artificial que é usada para sobrepor imagens ou vídeos existentes em outros criando, desta forma, uma informação adulterada que pode ser utilizada em âmbito político, criação de \textit{fakenews}, escandalos pornográficos com celebridades, etc.

O problema gerado pela disseminação de \textit{fake news} se torna mais impactante quando olhamos para temas mais sensíveis como, por exemplo, política ou o mercado financeiro. Nesse contexto a disseminação de notícias falsas pode acarretar diversos problemas. O problema da disceminação desenfreada de notícias falsas deve ser combatido, dado este fato, o Congresso Nacional instalou uma comissão parlamentar mista de inquérito (CPMI)  para apurar a divulgação de informações falsas. \footnote{\url{https://g1.globo.com/politica/noticia/2019/09/04/congresso-instala-comissao-para-investigar-divulgacao-de-informacoes-falsas.ghtml}} Com base neste fato, a utilização de sistemas de identificação de notícias falsas faz-se fundamental para redução da disseminação dessas fontes de desinformação.

% Primeiro parágrafo diz o que o artigo faz.

% Próximos parágrafos fazem uma introdução ao artigo.

\subsection{Motivação}

% Nessa parte deve ser dada uma motivação ao problema, descrevendo porque ele é importante.

% Deve ser indicado um ``gap'' que será investigado, se houver.

As notícias falsas que se espalham pelos meios de comunicação representam uma ameaça real à confiabilidade das informações.
A informação e a detecção de notícias falsas têm atraído cada vez mais atenção nos últimos anos. Notícias falsas
são normalmente escritas intencionalmente para enganar os leitores, o que determina que a detecção de notícias falsas
meramente baseado em conteúdo de notícias é tremendamente desafiador. Enquanto isso, as notícias falsas podem
conter evidências verdadeiras para zombar de notícias verdadeiras e apresentar diferentes graus de falsidade, o que ainda
agrava a dificuldade de detecção \cite{karimi2018}. \\

Segundo \cite{allcot2017}, que fizeram um estudo a respeito da disseminação de \textit{fake news} nas redes sociais durante as eleições presidenciais dos Estados Unidos de 2016, parecem existir duas motivações principais para disseminação de \textit{fake news}. A primeira é pecuniária: artigos de notícias que se tornam virais nas mídias sociais podem atrair anúncios significativos aumentar a receita quando os usuários clicam no site original. 
Esta parece ter sido a principal motivação para a maioria dos produtores cujas identidades foram reveladas. 
A segunda motivação é ideológica pois alguns provedores de notícias falsas procuram promover os candidatos que preferem. \\

Além de impactos sociais e políticos, as \textit{fake news} também podem impactar de forma significativa o mercado financeiro. Prova disso foi o ``Joesley Day'' \footnote{\url{https://www.shsinvestimentos.com.br/artigos-shs/fake-news-e-o-mercado-financeiro-entenda-como-elas-afetam-as-operacoes}} que aconteceu em 2017. Na ocasião, um áudio gravado pelo dono da JBS - que fazia parte de uma delação premiada - registrava o então presidente Michel Temer dando carta branca para comprar o silêncio do ex-deputado Eduardo Cunha, que estava preso por corrupção na operação Lava Jato. Os investidores se desesperaram com a possibilidade de renúncia do presidente e o mercado afundou com os dois pés. O áudio era fake? Não. Só que Joesley sabia o que tinha nas mãos e manipulou o vazamento dele pra ganhar no mercado financeiro através da compra antecipada de dólares. Ele comprou US\$ 3 bilhões e como a moeda disparou, teve um lucro na casa dos US\$ 100 milhões, enquanto muitos investidores agressivos simplesmente quebraram.

\subsection{Objetivo}

% Aqui deve ser dito ou o objetivo, ou as questões de pesquisa ou as hipóteses.

Este trabalho tem como objetivo fazer uma breve revisão da literatura a fim de identificar o estado da arte a respeito do assunto de detecção de notícias falsas, mais especificamente: (i) Quais são as abordagens mais utilizadas em termos de modelagem do problema (Problema de classificação ou regressão); (ii) Quais as técnicas de processamento de linguagem natural e os algoritmos de aprendizado de máquina que são mais eficazes para esta tarefa; (iii) e por fim, executar experimentos utilizando a rede neural \textbf{BERT} e realizar uma comparação com os métodos utilizados na literatura.

% \section{Comentários sobre este modelo}

% O objetivo deste modelo é seguir um padrão para o curso de Tópicos Especiais de Engenharia de Dados e Conhecimento. Ele é baseado na descrição do método científico feita por~\citep[p. 39-40]{Bunge2002}\footnote{Tradução livre do autor}:
% \begin{quote}
% \begin{enumerate}
%     \item Descobrimento do Problema ou lacuna em um conjunto de conhecimentos. Se o problema não está enunciado com clareza, se passa à etapa seguinte, se não, à subsequente;
%     \item Descrição precisa do problema, se possível em termos matemáticos, mas não necessariamente quantitativos, ou uma nova descrição de um velho problema a luz de novos conhecimentos 
%     \item Busca de conhecimentos ou instrumentos relevantes ao problema (por exemplo, dados empíricos, teorias, aparatos de medida, técnica de cálculo ou de medida). Ou seja, inspeção do conhecido para ver se é possível resolver o problema;
%     \item Tentativa de solução do problema com ajuda dos meios identificados. Se essa tentativa falha, passasse à etapa seguinte, se não, à subsequente.
%     \item Invenção de novas ideias (hipóteses, teorias ou técnicas) ou a produção de novos dados empíricos que prometam resolver o problema;
%     \item Obtenção uma solução, exata ou aproximada, do problema com auxílio do instrumental conceitual ou material disponível;
%     \item Por a prova a solução, por exemplo, com ensaios de laboratório ou de campo, e
%     \item Correções necessárias nas hipóteses ou técnicas, ou mesmo na formulação do problema original.
% \end{enumerate}
% \end{quote}

% Essa descrição, ilustrada na Figura \ref{fig:bunge}, é o que se chama na Engenharia de Software de um processo Linear ou em Cascata, mas é óbvio que isso é só uma abstração que facilita a descrição a nível epistemológico. A pesquisa científica é um processo de aprendizado constante, e muitas vezes é necessário, após uma etapa, voltar atrás, ou pular a frente, de forma a melhorar a compreensão do problema, das soluções possíveis, corrigir experimentos, etc.

% \begin{figure}[hbt]
%     \centering
%     \includesvg[width=0.8\textwidth]{Imagens/metodologiabunge}
%     \caption{Metodologia Científica segundo \citet{Bunge2002}, descrita em BPMN~\citep{omg2013bpmn}. Fonte: Do Autor}
%     \label{fig:bunge}
% \end{figure}

% O mesmo autor ainda diz que para que uma ideia seja considerada científica é necessário, mas não suficiente, que ela seja objetivamente testável com dados empíricos~\citep[p. 37]{Bunge2002}. 

% Ainda mais, \citet[p. 40]{Bunge2002} cita \citet{Kuhn1970}, que diz que a melhor forma de aprender a planejar e resolver problemas científicos não é estudar um manual de metodologia, mas estudar e imitar paradigmas ou modelos de investigações que tiveram êxito. \citet{Kuhn2018}, em seu posfácio de 1969, cita outro autor, Michael Polanyi,  defende o conhecimento tácito e diz que ele ``é aprendido fazendo ciência, ao invés de adquirindo regras para fazê-la''~\citep[p. 160]{Kuhn2018}\footnote{O que é um sinal de aviso sobre este texto!}.


% Alguns artigos podem ser tomados como exemplo e não seguem exatamente a estrutura deste modelo. Se você deseja seguir outra estrutura, discuta com o professor primeiro.

% \subsection{Leitura Adicional}

% \begin{itemize}
%     \item Um artigo razoavelmente bem escrito na área de jogos\footnote{\url{https://www.overleaf.com/read/vbhmybkwnmpf}}.
%     \item Não deixem de ler o \citetitle{xexeo2021}.
%     \item Artigos da revista \textit{Pattern Recognition Letters}\footnote{\url{https://www.sciencedirect.com/journal/pattern-recognition-letters}} estão razoavelmente alinhos com padrões de organização aceitos nessa cadeira.
%     \item Alguns alunos devem um trabalho  ainda não publicado ``\textit{Guidelines} para a Construção de uma Dissertação de Mestrado em Computação Aplicada com Problemas de  Mineração de Texto''\footnote{\url{https://www.overleaf.com/read/qpwtkfdssxzr}}.
% \end{itemize}

