\begin{abstract}
   A disseminação de notícias falsas representam uma ameaça real à confiabilidade das informações e a detecção dessas notícias tem sido um tema que tem atraído muita atenção nos últimos anos e sua identificação pode beneficiar a sociedade em diversos aspectos. Dado o alcance das plataformas de comunicação como Twitter, Facebook, Youtube e afins, disseminar um conhecimento ou uma informação torna-se instantâneo, e tal fato contribui para a disseminação rápida desse tipo de conteúdo. O fato da informação presente em uma notícia poder ter vários graus de veracidade, isto é, as informações presentes podem não ser completamente verdadeiras ou completamente falsas, torna a resolução do problema de detecção e \textit{fake news} bastante desafiadora.
   Neste trabalho, foi feita uma breve revisão da literatura a fim de identificar o estado da arte a respeito do assunto de detecção de notícias falsas, mais especificamente quais são as abordagens mais utilizadas em termos de modelagem do problema (Problema de classificação ou regressão), quais as técnicas de processamento de linguagem natural e os algoritmos de aprendizado de máquina que são mais eficazes para esta tarefa, e por fim, executar experimentos utilizando a rede neural \textbf{BERT} e realizar uma comparação com os métodos utilizados na literatura.
   
   
%     A maioria das publicações usa um resumo não estruturado, que será usado aqui, mas em todo caso, nós usaremos os mesmos tópicos propostos em uma estrutura chamada IMRD~\citep{cmucs}. O artigo, porém, não usará essa estrutura, mas a que aqui está descrita.
%     O resumo deve ser construído de forma a fornecer as seguintes informações: 
%     \begin{itemize}
%         \item introdução, dizendo o que é o artigo
%         \item métodos
%         \item resultados principais
%         \item uma discussão dos resultados, por exemplo, porque é importante
%     \end{itemize} 


\end{abstract}
