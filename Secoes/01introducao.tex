\section{Introdução}

Primeiro parágrafo diz o que o artigo faz.

Próximos parágrafos fazem uma introdução ao artigo.

\subsection{Motivação}

Nessa parte deve ser dada uma motivação ao problema, descrevendo porque ele é importante.

Deve ser indicado um ``gap'' que será investigado, se houver.

\subsection{Objetivo}

Aqui deve ser dito ou o objetivo, ou as questões de pesquisa ou as hipóteses.

\section{Comentários sobre este modelo}

O objetivo deste modelo é seguir um padrão para o curso de Tópicos Especiais de Engenharia de Dados e Conhecimento. Ele é baseado na descrição do método científico feita por~\citep[p. 39-40]{Bunge2002}\footnote{Tradução livre do autor}:
\begin{quote}
\begin{enumerate}
    \item Descobrimento do Problema ou lacuna em um conjunto de conhecimentos. Se o problema não está enunciado com clareza, se passa à etapa seguinte, se não, à subsequente;
    \item Descrição precisa do problema, se possível em termos matemáticos, mas não necessariamente quantitativos, ou uma nova descrição de um velho problema a luz de novos conhecimentos 
    \item Busca de conhecimentos ou instrumentos relevantes ao problema (por exemplo, dados empíricos, teorias, aparatos de medida, técnica de cálculo ou de medida). Ou seja, inspeção do conhecido para ver se é possível resolver o problema;
    \item Tentativa de solução do problema com ajuda dos meios identificados. Se essa tentativa falha, passasse à etapa seguinte, se não, à subsequente.
    \item Invenção de novas ideias (hipóteses, teorias ou técnicas) ou a produção de novos dados empíricos que prometam resolver o problema;
    \item Obtenção uma solução, exata ou aproximada, do problema com auxílio do instrumental conceitual ou material disponível;
    \item Por a prova a solução, por exemplo, com ensaios de laboratório ou de campo, e
    \item Correções necessárias nas hipóteses ou técnicas, ou mesmo na formulação do problema original.
\end{enumerate}
\end{quote}

Essa descrição, ilustrada na Figura \ref{fig:bunge}, é o que se chama na Engenharia de Software de um processo Linear ou em Cascata, mas é óbvio que isso é só uma abstração que facilita a descrição a nível epistemológico. A pesquisa científica é um processo de aprendizado constante, e muitas vezes é necessário, após uma etapa, voltar atrás, ou pular a frente, de forma a melhorar a compreensão do problema, das soluções possíveis, corrigir experimentos, etc.

\begin{figure}[hbt]
    \centering
    \includesvg[width=0.8\textwidth]{Imagens/metodologiabunge}
    \caption{Metodologia Científica segundo \citet{Bunge2002}, descrita em BPMN~\citep{omg2013bpmn}. Fonte: Do Autor}
    \label{fig:bunge}
\end{figure}

O mesmo autor ainda diz que para que uma ideia seja considerada científica é necessário, mas não suficiente, que ela seja objetivamente testável com dados empíricos~\citep[p. 37]{Bunge2002}. 

Ainda mais, \citet[p. 40]{Bunge2002} cita \citet{Kuhn1970}, que diz que a melhor forma de aprender a planejar e resolver problemas científicos não é estudar um manual de metodologia, mas estudar e imitar paradigmas ou modelos de investigações que tiveram êxito. \citet{Kuhn2018}, em seu posfácio de 1969, cita outro autor, Michael Polanyi,  defende o conhecimento tácito e diz que ele ``é aprendido fazendo ciência, ao invés de adquirindo regras para fazê-la''~\citep[p. 160]{Kuhn2018}\footnote{O que é um sinal de aviso sobre este texto!}.


Alguns artigos podem ser tomados como exemplo e não seguem exatamente a estrutura deste modelo. Se você deseja seguir outra estrutura, discuta com o professor primeiro.

\subsection{Leitura Adicional}

\begin{itemize}
    \item Um artigo razoavelmente bem escrito na área de jogos\footnote{\url{https://www.overleaf.com/read/vbhmybkwnmpf}}.
    \item Não deixem de ler o \citetitle{xexeo2021}.
    \item Artigos da revista \textit{Pattern Recognition Letters}\footnote{\url{https://www.sciencedirect.com/journal/pattern-recognition-letters}} estão razoavelmente alinhos com padrões de organização aceitos nessa cadeira.
    \item Alguns alunos devem um trabalho  ainda não publicado ``\textit{Guidelines} para a Construção de uma Dissertação de Mestrado em Computação Aplicada com Problemas de  Mineração de Texto''\footnote{\url{https://www.overleaf.com/read/qpwtkfdssxzr}}.
\end{itemize}

