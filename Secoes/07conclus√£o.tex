\section{Conclusão}

% Conclusão geral indicando como foram atingido os objetivos, quais as contribuições e trabalhos futuros.

Com base na revisão da literatura que foi realizada, foi descoberto que existem diversas técnicas de pré-processamento que são utilizadas para atacar o problema da detecção de \textit{fake news} como, por exemplo, remoção de \textit{stopwords}, TF-IDF, N-GRAM, etc. Entretanto este trabalho mostra o \textit{Google BERT} é uma poderosa ferramenta para classificação de \textit{fake news}, pois o seu uso sem o acompanhamento de técnicas de pré-processamento adicionais se mostrou extremamente efetivo para a tarefa obtendo um \textit{F1 score} de 0.991 e conseguindo um resultado melhor que os obtidos por diversos algoritmos como, \textit{SVM, LR, Naive Bayes, etc.}.


Como trabalhos futuros, podemos analisar o impacto do uso de \textit{features}, como por exemplo o autor da notícia bem como analisar o modelo proposto em outra base de dados, Dado que os desenvolvedores da base de dados que foi utilizada neste trabalho explicam que notícias que eram parcialmente verdadeiras não foram incluídas no \textit{Corpus}. 

