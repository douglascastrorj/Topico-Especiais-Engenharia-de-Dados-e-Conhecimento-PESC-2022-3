\section{Análise dos Resultados}

% Aqui serão analisados os resultados dos experimentos.

% Os resultados devem ser \textbf{interpretados} e não apenas listados. 

% Deve ser explicado o que os resultados significam para a área.

Como mencionado anteriormente, um dos objetivos deste trabalho é identificar as abordagens mais utilizadas para atacar o problema da detecção de \textit{fake news}. Com base na revisão da literatura feita por este trabalho, podemos citar como um dos resultados os seguintes achados: (1) As técnicas  de \textit{NLP} que se mostraram mais eficazes foram \textit{BoW, TF-IDF e N-GRAM} e os algoritmos que se sobressaíram foram \textit{SVM, LR} e algoritmos baseados em redes neurais ou \textit{Deep Learning}, principalmente os que consideram informações do usuário para a detecção de \textit{fake news};  (2) O sentimento da presente no conteúdo da notícia não auxilia na detecção de \textit{fake news}, pois como foi mostrado em \citep{baarir2020} o sentimento da notícia não influencia em sua veracidade; (3) A utilização de \textit{Stance Detection} também não ajuda na detecção de \textit{fake news} como mostrado em \citep{DeMagistris2022}.

Outro resultado deste trabalho foi a obtenção de uma taxa de acerto de 99\% na detecção de \textit{fake news} utilizando o \textit{Google BERT} no segundo experimento que foi superior ao do primeiro experimento,que obteve cerca de 93\% de acerto, que também utilizava o \textit{BERT} porém fazendo uso pré-processamento do texto. Esse resultado mostra que o \textit{Google BERT} funciona melhor utilizando o texto em linguagem natural. Também vale ressaltar que, nesses experimentos, não foram utilizadas informações a respeito do autor da notícia ou do usuário que a propagou oque indica que apenas o texto da notícia, para a base de dados utilizada, foi suficiente para a distinção entre notícias falsas e verdadeiras. 