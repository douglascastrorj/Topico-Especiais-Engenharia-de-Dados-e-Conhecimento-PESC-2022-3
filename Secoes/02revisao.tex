\section{Revisão}

\todo[inline]{Pedi várias coisas aqui, mas tem que colocar em uma ordem boa, e pode usar subseções. Por exemplo, Datasets, BERT e trabalhos prévios do grupo}

\todo[inline]{Falar sobre o BERT, com explicação dos conceitos de atenção, transformers e também do BERTIMBAU}

\todo[inline]{Defina rapidamente e de a referência para as medidas utilizadas}

\todo[inline]{Falar das bases que existem para teste e são bastante usadas Falar do trabalho de Silva et al. em português}

\begin{table}[hbt]
\centering
\caption{Frequência das notícias por tópico no Fake Br \citep{Silva2020}.}
\label{tab:topfakebr}
    \begin{tabular}{cc}
    \toprule
    Tópico & Frequência \\
    \midrule
    Sei lá & Quanto \\
    \bottomrule
    \end{tabular}
\end{table}

\todo[inline]{Falar do trabalho do Igor na COPPE, algo como "nosso grupo já trabalhou com detecção de fake news em língua inglesa"... }

Ver \url{https://www.cos.ufrj.br/index.php/pt-BR/publicacoes-pesquisa/details/15/2936}